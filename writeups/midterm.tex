\documentclass[11pt,preprint]{aastex}
 %\documentclass[12pt]{emulateapj}
\usepackage[margin= 1.0in]{geometry}    % See geometry.pdf to learn the layout options. There are lots.
\geometry{letterpaper} % or letter or a5paper or ... etc
\usepackage{float}
\usepackage{amssymb,amsmath}
\usepackage[]{epsfig,graphicx}
\usepackage{color}
\usepackage{verbatim}
\DeclareGraphicsRule{.tif}{png}{.jpg}{`convert Num1 `dirname Num1`/`basename Num1 .tif`.jpg}
\newcommand{\units}[1]{\ensuremath{\, \mathrm{#1}}}
\usepackage{enumitem}
\usepackage{natbib}
\newcommand{\degree}{\ensuremath{^\circ}}
\usepackage[maxfloats=25]{morefloats}
\usepackage[normalem]{ulem}
\usepackage{hyperref}
\usepackage{ amssymb }
\usepackage{lscape}
\newcommand{\TRANSPOSE}{\ensuremath{T}}

\bibliographystyle{apalike}


\begin{document}
\title{Cure Violence!}

 \author{Andy Enkeboll, Erin Grand, and Mayank Misra}
 \affil{Data Science Institute, Columbia University, New York, NY 10027}
 
\date{\today}             

%\begin{abstract}
%This is the abstract.
%\end{abstract}


\tableofcontents

\section{Introduction}
Cure Violence is a non-profit that focuses on stopping the spread of  gun violence in communities by using the methods and strategies associated with disease control, detecting and interrupting conflicts, identifying and treating the highest risk individuals, and changing social norms. Specifically, they focus on:
\begin{enumerate}
\item Detecting and interrupting potentially violent events by preventing retaliations and mediating conflicts through on-the-ground efforts within communities.
\item Assessing high risk candidates, changing behavior, and providing treatment through 1-on-1 case work.
\item Engaging the community to change by responding to every shooting, and organizing community events and people.
\end{enumerate}
Through partners at Booz Allen Hamilton, Cure Violence is seeking to make better use of the data they collect and the independent reports that have been written about them.  From the outset, Cure Violence have outlined four main goals they have for this capstone:
\begin{enumerate}
    \item Demonstrate how Cure Violence unique operating model can be deployed across different geographies to reduce violence and increase cost savings to stakeholders both direct and indirect within communities and cities of Cure Violence operations.
    \item Data analysis and/or model to inform growth of existing program services
    \item Data analysis and/or model to inform diversification of service (i.e., new areas of violence)
    \item Data to inform potential strategic partners & alliances
\end{enumerate}


\section{Plan / Milestones}
From the stated goals outlined above, a plan was created that more accurately represents what this Capstone project can do in terms of time and people:

\begin{enumerate}
\item Cost benefit analysis - simple:
How many Incidents were avoided, multiplied by the cost saving scalar,
Extract data from public reports
\item Cost benefit analysis - complex:
Comparison to national averages in Chicago/Baltimore,
Exploration of the government data,
Lift of incidents avoided
\item Joining and consolidating internal and external government data sets
\item Solid path forward for cure violence of where to go next from a cost benefit analysis perspective
\end{enumerate}


\section{Data Collections and Analysis}

We've looked at data of three different types. (1) Public record paper results from independent researchers in Chicago and Baltimore. (2) City and state open data such as 311 and 911 records. (3) the records that Cure Violence keeps as internal data.


\subsection*{Public Reports}
Three reports have been conducted by researchers in Chicago and Baltimore, at John's Hopkins,  University of Illinois: Chicago and Northwest University. Each study looked at how the Cure Violence programs were used and assessed the success of the programs. It was found that the Cure Violence framework was applied in Chicago and Baltimore neighborhoods with measurable success.   



\subsubsection*{Johns Hopkins University: Baltimore, 2012}

The John Hopkins team looked at research from two programs in Baltimore, (a) and (b). for each program there were 30-40 participants taking surveys and recording internal data. 

Data from the Baltimore Police Department for homicides and nonfatal shootings from January 1, 2003 to December 31, 2010
Surveys of program participants from 2007 - 2010


Three of the four program sites experienced large, statistically significant, program related reductions in homicides or nonfatal shootings without having a counter-balancing significant increase in one of these outcome measures.


\subsubsection*{UI Chicago: Chicago, 2009, Northwestern University: Chicago, 2014}

Raw crime counts showed a 31\% reduction in homicide, a 7\% reduction in total violent crime, and a 19\% reduction in shootings in the targeted districts. These effects are significantly greater than the effects expected given the declining trends in crime in the city as a whole.

The reduced levels of total violent crime, shootings, and homicides remained constant past the time frame of the survey analysis.  The effects of the intervention were seen immediately and therefore it is unlikely that effects were only due to increased police activity.

\subsection*{Open Data}
The open data we wish to look at will help toward suggesting a recommendation for Cure Violence to focus on next. 311 and 911 reports of crime and gun violence are useful for location information. Census/population data can be added in as well to determine how similar demographics of cities might be.


\subsection*{Proprietary Data}
Cure Violence itself keeps careful data from each intervention. There are:
\begin{enumerate}
\item Case notes (interventions)
\item Conflict mediation reports
\item Community Events \& participants
\item Crime data reported specifically in the bounds of the neighborhood
\end{enumerate}
Cure Violence has been very cautious about giving this data out, unfortunately. The fourth goal as stated in section 2 would require this data from Cure Violence in order to tie together the internal, proprietary data with the external, open data. 

\section{Model of Cost Savings}
The evaluations of Baltimore and Chicago focused on the correlation between application of the model and related killing and shooting incidences in the area. One of the key objectives of our analysis is to quantify this impact in financial terms.  

In an effort to estimate the dollar impact from gun crimes, we have used the cost categories and estimates described in Mother Jones analysis done by Mark Follman, Jeah Lee and Julia Lurie on the true cost of gun violence in America as a template to create a financial model.  This analysis was based on the research done by economist Ted Miller of Pacific Institute for Research and Evaluation.  

\subsection*{Model Assumptions}
Our intent with the cost model is to provide an estimate on the dollar impact to the societies where the Cure Violence model was applied.  We made the following decisions to tie the three primary sources on which our cost estimation model is based:  

To keep the model simple, we have rolled up cost categories as described in the Mother Jones article.  (1)	For direct costs, the 'Police' and 'Legal service and adjudication' categories have been clubbed under 'Police and Legal'.  The 'Medical' and 'Emergency transport' categories are aggregated under 'Medical care and Transport' in our model. (2)	For indirect costs, 'work cost of victims and perpetrators' and 'cost to employers' are aggregated under 'lost wages and productivity'.

We recognize that the way gun crime incidences are defined in the North Western, John Hopkins, and the Mother Jones studies will not match perfectly.

\begin{figure}[h]
    \centering
    \includegraphics[width=6in]{CureViolenceCostModel.png}
    \caption{Example Cost Model}
    \label{fig:my_label}
\end{figure}

The estimate cost for a particular category were derived, either directly as quoted in the Mother Jones article, or have been deciphered from the graphs accompanying the write up.  

\section{Next Steps}
For the remainder of the project we will be continuing to work on this model by (1) enhancing the model to measure the impact in Purchasing Power Parity (PPP) of the year in which the Cure Violence program was applied.   (2) Understand the Net Present Value (NPV) of the cost avoided. (3) A mapping of gun crime incidences between the studies is provided in the appendix section.

We also intend on covering the reminder of the milestones listed above.

\section{Challenges}
As mentioned above, the main problem so far has been with tempering the expectations of the Cure Violence team to more accurately match what the Columbia team is able and likely to achieve. On the one hand, it took several weeks to establish deliverables and a timeline. On the other hand, Cure Violence didn't reveal until several weeks in that they were uncomfortable sharing their internal data sets. These challenges are certainly not insurmountable, as the project milestone shows. All that is needed to accomplish these goals is available to the Columbia team, and the guidance from Booz Allen and Columbia has helped to mitigate and move the project forward.

\newpage
\bibliography{msdref}

\end{document} 
